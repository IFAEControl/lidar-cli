\documentclass[letterpaper, 10 pt]{article}
\pagenumbering{Roman}

\usepackage[margin=1.00in]{geometry}
\usepackage{color}
\usepackage{hyperref}
\usepackage{graphicx}
\usepackage{authblk}
\usepackage{parskip}
\usepackage{listings}
\usepackage[dvipsnames]{xcolor}
\usepackage[english]{babel}
\usepackage{framed}
\usepackage{indentfirst}

\setlength{\parindent}{0.5cm}

\makeatother

\hypersetup{
    colorlinks,
    linktoc=all,
    citecolor=black,
    filecolor=black,
    linkcolor=black,
    urlcolor=black
}

\makeatletter

\definecolor{light-gray}{gray}{0.88}
\lstset{ 
			backgroundcolor=\color{light-gray},
			xleftmargin=.25in,
} 

% switch to key style at EOL
\lst@AddToHook{EveryLine}{\ifx\lst@language\language@yaml\YAMLkeystyle\fi}
\makeatother

\newcommand\ProcessThreeDashes{\llap{\color{cyan}\mdseries-{-}-}}
\newcommand{\cmdsection}[1]{\paragraph{#1}\mbox{}\par}

\begin{document}
\title{LIDAR CLI User Manual - 1.2.0}
\author{David Roman \\  \href{mailto:droman@ifae.es}{droman@ifae.es} }
\affil{IFAE}
\maketitle
\footnote{All commands referenced in this manual were tested using an Ubuntu 18.04.5 LTS version. }
\pagestyle{empty}
\newpage
\tableofcontents
\newpage
\pagenumbering{arabic}
\pagestyle{plain}

\section{Running licli}
\subsection{First steps}
If it's the first time we run licli it would ask us to introduce a password. This password is defined in the configuration file of the lidar server software. 
In case that it's not the first time and jwt secret has changed we can have some errors like \textbf{"The Token's Signature resulted invalid when verified using the Algorithm"}. If it's the case remove \textit{\~{}/.cache/lidar/token} and try again.\\

\subsection{Cheat sheet}
\subsubsection{Taking data}

\begin{enumerate}
	\item Startup everything\\
	 > licli operation -{}-startup\\
	 
	Open/Close only the container doors \\
		> licli motors doors --open\\
		> licli motors doors --close\\
		
	\item Remove the black plastic cap from the liquid light guide at the focus \\
	\item Move telescope to max. zenith\\
	> licli operation telescope --to-max-zenith\\
	
	Go to another zenith angle\\
	> licli operation telescope --go-zenith=XX       (XX=zenith angle in deg ) \\
	\item Check for airplanes at: \url{https://www.flightradar24.com/41.52,2.07/11}\\
	\item Set the laser power\\
	> licli llc laser --power=XX    (X= percent of laser power, from 0 to 100)\\
	\item Shoot the laser\\
	> licli llc laser --fire\\
	\item Take data (remember that the laser must be firing to take data) \\
	> licli operation acq --shots=500 --disc=1:2:1:8\\
	\item Pause the laser\\
	> licli llc laser --pause\\
	\item Shutdown everything\\
	> licli operation --shutdown 
\end{enumerate}

\subsection{Examples}
Open doors: \textbf{./licli motors doors \texttt{-{}-}open}

Get converted sensors values: \textbf{./licli llc sensors}

Make 4 shots with disc level 2: \textbf{./licli operation acq \texttt{-{}-}shots=4 \texttt{-{}-}disc=2:2:2:2}. Output data will be in \textit{\~{}/.local/share/lidar-cli}.

\subsection{Start up}
To start up the system in the standard way execute the "startup" command of \ref{operationcmds}.\\

If for some reason this fails we can do it manually with the next commands:
\begin{itemize}
	\item \textbf{./licli motors doors \texttt{-{}-}open}
	\item \textbf{./licli motors telescope \texttt{-{}-}sz=600}
	\item \textbf{./licli operation ll  \texttt{-{}-}micro-init}
	\item \textbf{./licli motors petals \texttt{-{}-}open}
	\item \textbf{./licli llc laser \texttt{-{}-}init}
	\item  \textbf{./licli llc laser \texttt{-{}-}power=5}
\end{itemize}
\subsection{Data acquisition}
When an acquisition is made the server stores the data in the licel format and returns an ID, this file can be afterwards downloaded using that identifier. For example to make an acquisition with discriminator levels 1,2,1 and 10 and 100 shots:\\
\textbf{./licli operation acq \texttt{-{}-}shots=100 \texttt{-{}-}disc=1:2:1:10}  (this will return the file ID, 1234 for example)
\textbf{./licli operation acq \texttt{-{}-}download=1234 \texttt{-{}-}disc=1:2:1:10}
\subsection{Shutdown}
To shutdown the system in the standard way execute the "shutdown" command (section \ref{operationcmds} briefly describes what it does). \\

\subsection{Warm up}
The command for the warm up is described in section \ref{operationcmds}.

\subsection{Connect to internal network}
There are two switches. One of them is used for all the devices that are accessible from outside the container, the other contains all devices connected to the internal network. To access this network connect a laptop with an ethernet cable to this internal switch with 255.255.255.0 as the mask and an IP equal or below 192.168.127.230. Inside this local network the server is at 192.168.127.253.

\section{Commands reference}
\subsection{LLC}
This command contains all the commands related with the low level control board.
\subsubsection{Arms}
\cmdsection{Synopsis} \textbf{licli llc arms [OPTION]}\\

\cmdsection{Description} Command to control and monitor laser arm. \\
\cmdsection{Options}
\begin{itemize}
	\item[] \textbf{-{}-check-node=N} Check communication with node N (N must be either "1" or "2")
	\item[] \textbf{-{}-emergency-stop} Execute emergency stop
	\item[] \textbf{-{}-get-pos} Get arm current position
	\item[] \textbf{-{}-go=X:Y} Go to current position. Argument must follow the format X:Y (eg: 1000:10000)
	\item[] \textbf{-{}-init} Initialize arm
	\item[] \textbf{-{}-set-speed=axis:speed} Set speed. Argument must follow the format Axis:speed.
\end{itemize}
\subsubsection{DAC}
\addtolength{\leftskip}{10 mm}
\cmdsection{Synopsis} \textbf{licli llc dac [OPTION] \\}

\cmdsection{Description} Command for controlling DAC voltages.\\
\cmdsection{Options}
\begin{itemize}
	\item[] \textbf{-{}-set-voltage=idx:voltage} Set DAC voltage. \textbf{idx} is the index of the desired DAC. Voltage is in mV. Setting a voltage of 1000 mV will produce a reading of 1 Volt, but note that the final voltage which will be multiplied by 1000 Volts by the PMT. 
\end{itemize}
\addtolength{\leftskip}{-10 mm}

\subsubsection{Drivers}
\cmdsection{Synopsis} \textbf{licli llc drivers [OPTION] \\}

\cmdsection{Description} Get information about the drivers.\\
\cmdsection{Options}
\begin{itemize}
	\item[] \textbf{-{}-status} Get a list of drivers status
\end{itemize}

\subsubsection{Hotwind}
\cmdsection{Synopsis} \textbf{licli llc hotwind [OPTION]}\\

\cmdsection{Description} Hotwind control.\\
\cmdsection{Options}
\begin{itemize}
	\item[] \textbf{-{}-error} Set error ¿?
	\item[] \textbf{-{}-lock} Lock
	\item[] \textbf{-{}-unlock} Unlock
\end{itemize}
\subsubsection{Laser}
\cmdsection{Synopsis} \textbf{licli llc laser [OPTION]}\\

\cmdsection{Description} Laser control.\\
\cmdsection{Options}
\begin{itemize}
	\item[] \textbf{-{}-fire} Fire the laser
	\item[] \textbf{-{}-fire-single} Fire a single shot
	\item[] \textbf{-{}-get-temp} Get laser temperature.
	\item[] \textbf{-{}-get-counter} Get shots counter.
	\item[] \textbf{-{}-init} Initialize laser
	\item[] \textbf{-{}-pause} Pause the laser
	\item[] \textbf{-{}-power} Set power in \%
	\item[] \textbf{-{}-power-us} Set power in microseconds.
	\item[] \textbf{-{}-stop} Stop the laser.
\end{itemize}

\subsubsection{Relays}
\cmdsection{Synopsis} \textbf{licli llc relays [OPTION]}\\

\cmdsection{Description} Commands to switch on/off relays.\\
\cmdsection{Options}
\begin{itemize}
	\item[] \textbf{-{}-status} Get relays status
	\item[] \textbf{-{}-hotwind-off} Disable hotwind
	\item[] \textbf{-{}-hotwind-on} Enable hotwind. When the temperature reaches the target it will be stopped.
	\item[] \textbf{-{}-laser-on} Enable laser
	\item[] \textbf{-{}-laser-off} Disable laser
	\item[] \textbf{-{}-licel-on} Enable laser
	\item[] \textbf{-{}-licel-off} Disable laser
\end{itemize}

\subsubsection{Sensors}
\cmdsection{Synopsis} \textbf{licli llc sensors [OPTION]}\\

\cmdsection{Description} Commands to get sensor information from low level control board. There are other sensors that this command doesn't read, for example the temperature of the head of the laser.\\
If no option is specified it will print values in human readable form.\\
\cmdsection{Options}
\begin{itemize}
	\item[] \textbf{-{}-raw} Get raw sensor values
\end{itemize}

\subsection{Motors}
\subsubsection{Doors}
\cmdsection{Synopsis} \textbf{licli motors doors [OPTION]}\\

\cmdsection{Description} Commands to control motor doors \\
\cmdsection{Options}
\begin{itemize}
	\item[] \textbf{-{}-close} Close doors.
	\item[] \textbf{-{}-open} Open doors
	\item[] \textbf{-{}-stop} Stop doors. 
	\item[] \textbf{-{}-status} Print current status of the doors (OPEN, CLOSE or INTERSTATE if it's something in the middle)
\end{itemize}
\subsubsection{Petals}
\cmdsection{Synopsis} \textbf{licli motors petals [OPTION]}\\

\cmdsection{Description} Commands to control petals doors.\\
\cmdsection{Options}
\begin{itemize}
	\item[] \textbf{-{}-close} Close petals.
	\item[] \textbf{-{}-open} Open petals
	\item[] \textbf{-{}-stop} Stop petals. 
	\item[] \textbf{-{}-status} Print current status of the petals (CLOSED or UNKNOWN if there is something else)
\end{itemize}
\subsubsection{Telescope}
\cmdsection{Synopsis} \textbf{licli motors telescope [OPTION]}\\

\cmdsection{Description} Commands to control telescope motors. \\
\cmdsection{Options}
\begin{itemize}
	\item[] \textbf{-{}-ga} Get current azimuth position
	\item[] \textbf{-{}-gz} Get current zenith position
	\item[] \textbf{-{}-home} Go home. If tries to go to the position registered before opening the doors, if it's not possible it defaults to the hardcoded values, which may not be the correct ones. For this reason it's important to start the server when the doors are closed.
	\item[] \textbf{-{}-sa} Go to the given azimuth encoder position (an unsigned integer)
	\item[] \textbf{-{}-sz} Go to the given zenith encoder position (an unsigned integer)
\end{itemize}
\subsection{Monitoring}
Monitoring commands
\subsubsection{Sensors}
\cmdsection{Synopsis} \textbf{licli monitoring sensors [OPTION]}\\

\cmdsection{Description} Monitoring of sensors.\\
\cmdsection{Options}
\begin{itemize}
	\item[] \textbf{-{}-humidity} Show readings of humidity values 
	\item[] \textbf{-{}-env-temperature} Show readings of environment temperatures
	\item[] \textbf{-{}-last-value} (can be used in combination): Show the last value reading
\end{itemize}
\subsubsection{Motors}
\cmdsection{Synopsis} \textbf{licli monitoring motors}\\

\cmdsection{Description} Monitoring of the motor monitoring board. It can show encoders position or a bitmap of the current status.

\subsection{Alarms}
\cmdsection{Synopsis} \textbf{licli monitoring alarms}\\

\cmdsection{Description} Waits for alarm messages. Alarms are triggered whenever a sensor exceeds the safe limit. Message are sent in the form of strings informing of the problem.

\subsection{Config}
\cmdsection{Synopsis} \textbf{licli monitoring config [OPTION]}\\

\cmdsection{Description} Config related operations. \\
\cmdsection{Options}
\begin{itemize}
	\item[] \textbf{-{}-get} Returns current configuration of the server.
	\item[] \textbf{-{}-check} Check client configuration. If no errors are found it will print \emph{Configuration OK}.
\end{itemize}

\subsection{Trace}
\cmdsection{Description} Command for debugging purposes

\subsection{Operation} \label{operationcmds}
\cmdsection{Synopsis} \textbf{licli operation [OPTION]}\\

\cmdsection{Description} Unless other commands, operation commands (\ref{operationcmds}) are complex commands. Those complex commands can be macros (ie: a couple of commands executed sequentially) or control functions which makes use of the other functions. \\
\cmdsection{Options}
\begin{itemize}
	\item[] \textbf{-{}-startup} Prepare the laser to shoot. Open the doors, power on all subsystems and move telescope to a good position (ie:above tilt).
	\item[] \textbf{-{}-shutdown} Power off all subsystems, close petals, move the telescope to parking position and close the doors.
	\item[] \textbf{-{}-warmup} Start warm up. Enable hotwind and wait until it gets to the target temperature.
\end{itemize}
\subsubsection{Acquisition}
\cmdsection{Synopsis} \textbf{licli operation acq [OPTION]}\\

\cmdsection{Description} The acquired data is first stored in the server in licel file format. The server will return a file ID so we can download that file at anytime. Files are also downloaded automatically so we don't need to download it manually.\\
\cmdsection{Options}
\begin{itemize}
	\item[] \textbf{-{}-disc=A:B..} Discriminator levels. Always required. It's an array of discriminators levels separated by a ":". First value corresponds to the first discriminator and so.
	\item[] \textbf{-{}-download=N} Download licel file. Argument is the ID (unsigned integer) of the licel file we want to download.
	\item[] \textbf{-{}-shots=N} Acquire a given number of shots. Argument is an unsigned integer greater than 1. 
	\item[] \textbf{-{}-start} Start acquisition manually
	\item[] \textbf{-{}-stop} Stop acquisition
	\item[] \textbf{-{}-analog-data} Save a file with all analog values using space character as separator.
	\item[] \textbf{-{}-analog-combined} Save a file with normalized analog values.
	\item[] \textbf{-{}-raw-analog} Save lsw and msw arrays in raw format.
	\item[] \textbf{-{}-raw-squared} Save squared data in raw format.
\end{itemize}
\subsubsection{Telescope}
\cmdsection{Synopsis} \textbf{licli operation telescope [OPTION]}\\

\cmdsection{Description} Telescope motors operations. When going to parking position it will first try to go to the last know good value (ie: the position we know that doors can be closed), if it's not possible it will go to the hardcoded values.\\
\cmdsection{Options}
\begin{itemize}
	\item[] \textbf{-{}-get-azimuth} Get azimuth position
	\item[] \textbf{-{}-go-azimuth-parking} Go to azimuth parking position
	\item[] \textbf{-{}-go-parking} Go to parking position
	\item[] \textbf{-{}-go-zenith-parking} Go zenith parking position
	\item[] \textbf{-{}-go-zenith=N} Go to zenith inclination in degrees
	\item[] \textbf{-{}-start-test} Execute telescope tests (move azmiuth and zenith to their maximum, minimum and parking position).
	\item[] \textbf{-{}-stop} Stop any telescope movement
	\item[] \textbf{-{}-to-max-azimuth} Go to maximum azimuth
	\item[] \textbf{-{}-to-max-zenith} Go to maximum zenith
	\item[] \textbf{-{}-to-min-azimuth} Go to minimum azimuth		
\end{itemize}
\subsubsection{Low level}
\cmdsection{Synopsis} \textbf{licli operation ll [OPTION]}\\

\cmdsection{Description} Low level operations.\\
\cmdsection{Options}
\begin{itemize}
	\item[] \textbf{-{}-arm-align} Move arm to alignment position (first it must be initialized)
	\item[] \textbf{-{}-arm-init} Initialize arm
	\item[] \textbf{-{}-laser-init} Initialize laser (first it must be powered-on)
	\item[] \textbf{-{}-laser-power-off} Power off laser
	\item[] \textbf{-{}-laser-power-on} Power on laser
	\item[] \textbf{-{}-laser-fire} Fire the laser
	\item[] \textbf{-{}-micro-init} Micro initialization sequence. Prepare laser, arm and rump up all DACs.
	\item[] \textbf{-{}-micro-shutdown} Micro shutdown sequence. Power off laser hotwind and ramp down DACs
	\item[] \textbf{-{}-ramp-down} Ramp down all DACs
	\item[] \textbf{-{}-ramp-single} Modify voltage of a single DAC. Argument must follow the format idx:voltage.
	\item[] \textbf{-{}-ramp-up} Ramp up all DACs (defaults to 150)
\end{itemize}

\section{Configuration} \label{cfg}
Configuration files can be found on \textit{\~{}/.config/lidar/client/}. After modifying configuration files execute \textbf{licli config -{}-check} to ensure there are not syntax mistakes.
\begin{itemize}
	\item acquisition.properties: This config file is mandatory. It have four fields named wavelength\_ch\_N where N goes from 1 to 4. This specifies the wavelength for each channel. 
	\item micro\_init\_sequence.properties: Is used to defines arm alignment position, PMT voltages and temperature threshold.	\\
	
	Example: 
	\begin{verbatim}
temperature_threshold = 123

# PMT 355, Licel channel 1
pmt_dac_vlts_0 = 600
# PMT 607, Licel channel ??
pmt_dac_vlts_1 = 1500
# PMT 532, Licel channel 3
pmt_dac_vlts_2 = 600
# PMT 387, Licel channel 2
pmt_dac_vlts_3 = 1500
# near range
pmt_dac_vlts_4 = 1500
# not used
pmt_dac_vlts_5 = 1500

allignment_arm_X = 1610000
allignment_arm_Y = 960000
	\end{verbatim}
	\item networking.properties: Can be used to specify address and TCP port of the server. By default it connects to \textbf{lidar.ifae.es}.
\end{itemize} 
\subsection{Environment variables}
\begin{itemize}
	\item LIDAR\_INSECURE: By default licli uses tls encryption. If this variable is set to true licli won't use any encryption method. Note that the server must also allow plain text connections.
	\item LIDAR\_VERBOSITY: Increase verbosity level. Valid levels are 1,2,3, and 4.
\end{itemize}

\section{Troubleshooting}
In case of unknown problems, increase verbosity level to gather more information. To do this define \emph{LIDAR\_VERBOSITY} environment variable with an integer in the range of [1, 4], then execute licli again.
\subsection{The Token's Signature resulted invalid...}
Once we introduce the correct password a token is generated and save in our machine. This token allows us to execute new commands without having to introduce the password each time. Next time we execute a command the token (which is an encrypted message) will be sent to the server, validated and, if valid, it will execute our command. However, our local token may be invalid for multiple reasons (e.g: server password changed). The fix should be straightforward; remove \textit{\~{}/.cache/lidar/token} and try again. This way a new (valid) token will be generated.\\
\subsection{This method is not safe/read-only}
This means that the keylock is on which means that only those commands that don't produce side effects are allowed (ie: read-only commands) for example to read the status, etc. To fix it someone have to go physically and switch off the keylock.\\
Note: \textbf{enabling the keylock doesn't kill current running commands}
\subsection{Connection problems}
Execute \textbf{ping lidar.ifae.es} and see if there is an answer. The server may take up to 10 minutes to be completely operational, if after this time the the server doesn't answer ping requests, power off and on the lidar.

If the server is responding to ping requests, try to check if lidar service is also running. To do it check if port 81 is open (you can use a command like  \textbf{nc -vz lidar.ifae.es 81} to check it). If it isn't, power off and on the lidar.
\subsection{Where is X,Y,Z file/directory}
This program follow the XDG Base Directory specification. The full specification may be found on \url{https://standards.freedesktop.org/basedir-spec/basedir-spec-latest.html}\\
\linebreak
In short
\begin{itemize}
	\item \textit{ \~{}/.config/lidar}: Were config files are stored. See section \ref{cfg}.
	\item \textit{\~{}/.local/share/lidar}: Data directory. Licel, raw files and all related lidar output is stored here. 
	\item \textit{\~{}/.cache/lidar}: Cache dir. Only the token is stored there.
\end{itemize}

\newpage

%\bibliography{}

\end{document}